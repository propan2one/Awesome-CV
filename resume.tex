%!TEX TS-program = xelatex
%!TEX encoding = UTF-8 Unicode
% Awesome CV LaTeX Template for CV/Resume
%
% This template has been downloaded from:
% https://github.com/posquit0/Awesome-CV
%
% Author:
% Claud D. Park <posquit0.bj@gmail.com>
% http://www.posquit0.com
%
% Template license:
% CC BY-SA 4.0 (https://creativecommons.org/licenses/by-sa/4.0/)
%
% # To modified PDF to PNG to make some post prod modification
% pdftoppm {input.pdf} {output.file} -png

%-------------------------------------------------------------------------------
% CONFIGURATIONS
%-------------------------------------------------------------------------------
% A4 paper size by default, use 'letterpaper' for US letter
\documentclass[11pt, a4paper]{awesome-cv}

% Configure page margins with geometry
\geometry{left=1.4cm, top=.8cm, right=1.4cm, bottom=1.8cm, footskip=.5cm}

% Specify the location of the included fonts
\fontdir[fonts/]

% Color for highlights
% Awesome Colors: awesome-emerald, awesome-skyblue, awesome-red, awesome-pink, awesome-orange
%                 awesome-nephritis, awesome-concrete, awesome-darknight
%\colorlet{awesome}{awesome-skyblue}
% Uncomment if you would like to specify your own color
\definecolor{awesome}{HTML}{800080}

% Colors for text
% Uncomment if you would like to specify your own color
% \definecolor{darktext}{HTML}{414141}
% \definecolor{text}{HTML}{333333}
% \definecolor{graytext}{HTML}{5D5D5D}
% \definecolor{lighttext}{HTML}{999999}

% Set false if you don't want to highlight section with awesome color
\setbool{acvSectionColorHighlight}{true}

% If you would like to change the social information separator from a pipe (|) to something else
\renewcommand{\acvHeaderSocialSep}{\quad\textbar\quad}


%-------------------------------------------------------------------------------
%	PERSONAL INFORMATION
%	Comment any of the lines below if they are not required
%-------------------------------------------------------------------------------
% Available options: circle|rectangle,edge/noedge,left/right
% \photo[rectangle,edge,right]{./examples/profile}
\name{Jean}{Delmotte}
\position{Bioinformatician{\enskip\cdotp\enskip}Computational Biologist{\enskip\cdotp\enskip}Immunologist}
\address{285 rue André Philip - 69003 Lyon (France)}

\email{jeandelmotte0392@gmail.com}
%\homepage{www.jdlab.fr/}
\github{propan2one}
\mobile{(+33) 06-8721-5961}
% \linkedin{jean-delmotte-85311b82}
% \gitlab{gitlab-id}
% \stackoverflow{SO-id}{SO-name}
\twitter{@DrStagiaire}
% \skype{skype-id}
% \reddit{reddit-id}
% \medium{madium-id}
% \googlescholar{googlescholar-id}{name-to-display}
%% \firstname and \lastname will be used
\googlescholar{i1yRVZ4AAAAJ&hl=fr}{Jean Delmotte}
% \extrainfo{extra informations}

%-------------------------------------------------------------------------------
\begin{document}

% Print the header with above personal informations
% Give optional argument to change alignment(C: center, L: left, R: right)
\makecvheader[C]

% Print the footer with 3 arguments(<left>, <center>, <right>)
% Leave any of these blank if they are not needed
\makecvfooter
  {\today}
  {Jean Delmotte~~~·~~~Curriculum Vitae}
  {\thepage}


%-------------------------------------------------------------------------------
%	CV/RESUME CONTENT
%	Each section is imported separately, open each file in turn to modify content
%-------------------------------------------------------------------------------
%-------------------------------------------------------------------------------
%	SECTION TITLE
%-------------------------------------------------------------------------------
\cvsection{Summary}


%-------------------------------------------------------------------------------
%	CONTENT
%-------------------------------------------------------------------------------
\begin{cvparagraph}

%---------------------------------------------------------
Hi! My name is Jean Delmotte, I'm a PhD student working in Montpellier university (France). I am interested in understanding the infectious processes during host and a pathogen interaction. With my multidisciplinary background, I develop a scientific project in biology and bioinformatic including single-molecule long-read RNA and DNA sequencing data to help expand our knowledge a little more on infectious diseases.
\end{cvparagraph}

%-------------------------------------------------------------------------------
%	SECTION TITLE
%-------------------------------------------------------------------------------
\cvsection{Work Experience}


%-------------------------------------------------------------------------------
%	CONTENT
%-------------------------------------------------------------------------------
\begin{cventries}

%---------------------------------------------------------
  \cventry
    {\href{http://ihpe.univ-perp.fr/en/}{Host-Pathogen-Environment Interactions}, UMR-5244 - PhD student} % Job title
    {\href{}{Phylogeography, genetic diversity and connectivity of the Ostreid Herpesvirus 1 population in France}} % Organization
    {University of Montpellier, France} % Location
    {work in progress} % Date(s)
    {
      \begin{cvitems} % Description(s) of tasks/responsibilities
        \item {Bioinformatics analysis by writing the analysis pipeline, implementing all data visualizations and interpretation of results.}
        \item {First draft of the whole manuscript.}
      \end{cvitems}
    }

%---------------------------------------------------------
  \cventry
    {\href{http://ihpe.univ-perp.fr/en/}{Host-Pathogen-Environment Interactions}, UMR-5244 - PhD student} % Job title
    {\href{https://doi.org/10.3389/fmicb.2020.01579}{Contribution of Viral Genomic Diversity to Oyster Susceptibility in the Pacific Oyster Mortality Syndrome}} % Organization
    {University of Montpellier, France} % Location
    {July 2020} % Date(s)
    {
      \begin{cvitems} % Description(s) of tasks/responsibilities
        \item {Bioinformatics analysis by writing the analysis pipeline, implementing all data visualizations and interpretation of results.}
        \item {First draft of the whole manuscript.}
        \item {\href{https://doi.org/10.3389/fmicb.2020.01579}{Front. Microbiol., 10 July 2020: doi.org/10.3389/fmicb.2020.01579}}
      \end{cvitems}
    }

%---------------------------------------------------------
  \cventry
    {\href{https://www.institutcochin.fr/departments/3i/team-lucas/physiology-of-regulatory-foxp3-cd4-t-cells}{Physiology of regulatory Foxp3+ CD4 T cells}, U1016 - Master Student}% Job title
    {Regulation of effector functions of T lymphocytes: From basic research to cancer} % Organization
    {COCHIN Institute (PARIS), France} % Location
    {Oct. 2016 - Jun. 2017} % Date(s)
    {
      \begin{cvitems} % Description(s) of tasks/responsibilities
        \item {Study the circulation of lymphocytes in secondary lymphoid organs in a mouse model at homeostasis.}
        \item {Master thesis}
      \end{cvitems}
    }

%---------------------------------------------------------
  \cventry
    {ADVANCED IMMUNOLOGY - LabEx Milieu Interieur - Master student} % Job title
    {Characterization of the Immunological situation of healthy individuals using cutting edge technology} % Organization
    {PASTEUR Institute (PARIS), France} % Location
    {Nov. 2016 - Jan. 2017} % Date(s)
    {
      \begin{cvitems} % Description(s) of tasks/responsibilities
        \item {Realization and analysis of RNAseq experiments (IonTorrent), Biomark (Single Cell), ELISA multiplex, SeaHorse (metabolic test)}
        \item {Theoretical lessons on these techniques and on advanced immunology.}
      \end{cvitems}
    }

%---------------------------------------------------------
  \cventry
    {Génétique des virus et Pathogénèse des Maladies Virales, UMR-941 - Bachelor Student} % Job title
    {Inhibition of TNPO3 importin expression in HeLa cells by RNA interference} % Organization
    {Saint Louis hospital (PARIS), France} % Location
    {Apr. 2014 - Jun. 2014} % Date(s)
    {
      \begin{cvitems} % Description(s) of tasks/responsibilities
        \item {Carry out transfections using lentiviral vector coding an shRNA targeting TNPO3 protein on HeLa cells.}
        \item {Checked the inhibition effectiveness on nuclear import of HIV.}
      \end{cvitems}
    }

%---------------------------------------------------------
  \cventry
    {Technical platform} % Job title
    {Biological analysis technician} % Organization
    {Eurofins Biomnis (Ivry-sur-Seine), France} % Location
    {Jun. 2015 - Sep. 2015} % Date(s)
    {
      \begin{cvitems} % Description(s) of tasks/responsibilities
        \item {Carry out serological analysis on human samples}
      \end{cvitems}
    }
    
%---------------------------------------------------------
\end{cventries}

%%-------------------------------------------------------------------------------
%%	SECTION TITLE
%%-------------------------------------------------------------------------------
%\cvsection{Honors \& Awards}
%
%
%%-------------------------------------------------------------------------------
%%	SUBSECTION TITLE
%%-------------------------------------------------------------------------------
%\cvsubsection{International}
%
%
%%-------------------------------------------------------------------------------
%%	CONTENT
%%-------------------------------------------------------------------------------
%\begin{cvhonors}
%
%%---------------------------------------------------------
%  \cvhonor
%    {Finalist} % Award
%    {DEFCON 26th CTF Hacking Competition World Final} % Event
%    {Las Vegas, U.S.A} % Location
%    {2018} % Date(s)
%
%%---------------------------------------------------------
%  \cvhonor
%    {Finalist} % Award
%    {DEFCON 25th CTF Hacking Competition World Final} % Event
%    {Las Vegas, U.S.A} % Location
%    {2017} % Date(s)
%
%%---------------------------------------------------------
%  \cvhonor
%    {Finalist} % Award
%    {DEFCON 22nd CTF Hacking Competition World Final} % Event
%    {Las Vegas, U.S.A} % Location
%    {2014} % Date(s)
%
%%---------------------------------------------------------
%  \cvhonor
%    {Finalist} % Award
%    {DEFCON 21st CTF Hacking Competition World Final} % Event
%    {Las Vegas, U.S.A} % Location
%    {2013} % Date(s)
%
%%---------------------------------------------------------
%  \cvhonor
%    {Finalist} % Award
%    {DEFCON 19th CTF Hacking Competition World Final} % Event
%    {Las Vegas, U.S.A} % Location
%    {2011} % Date(s)
%
%%---------------------------------------------------------
%\end{cvhonors}
%
%
%%-------------------------------------------------------------------------------
%%	SUBSECTION TITLE
%%-------------------------------------------------------------------------------
%\cvsubsection{Domestic}
%
%
%%-------------------------------------------------------------------------------
%%	CONTENT
%%-------------------------------------------------------------------------------
%\begin{cvhonors}
%
%%---------------------------------------------------------
%  \cvhonor
%    {3rd Place} % Award
%    {WITHCON Hacking Competition Final} % Event
%    {Seoul, S.Korea} % Location
%    {2015} % Date(s)
%
%%---------------------------------------------------------
%  \cvhonor
%    {Silver Prize} % Award
%    {KISA HDCON Hacking Competition Final} % Event
%    {Seoul, S.Korea} % Location
%    {2017} % Date(s)
%
%%---------------------------------------------------------
%  \cvhonor
%    {Silver Prize} % Award
%    {KISA HDCON Hacking Competition Final} % Event
%    {Seoul, S.Korea} % Location
%    {2013} % Date(s)
%
%%---------------------------------------------------------
%\end{cvhonors}

%-------------------------------------------------------------------------------
%	SECTION TITLE
%-------------------------------------------------------------------------------
\cvsection{Presentation}


%-------------------------------------------------------------------------------
%	CONTENT
%-------------------------------------------------------------------------------
\begin{cventries}

%---------------------------------------------------------
  \cventry
    {Presenter for <Characterisation of OsHV-1 µVar genotypic diversity during POMS outbreaks in two French aquaculture area>} % Role
    {\href{https://eafp.org/19-eafp-porto-2019/}{19th International Conference on Diseases of Fish and Shellfish}} % Event
    {Porto, Portugal} % Location
    {Sep. 2019} % Date(s)
    {
      \begin{cvitems} % Description(s)
        \item {Introduced the history of OsHV-1 genomic diversity.}
        \item {Introduced how to evaluate the heterogeneity of viral populations}
      \end{cvitems}
    }

%---------------------------------------------------------
  \cventry
    {Presenter for <BibliographeR : a set of tools to help your bibliographic research>  - Text mining} % Role
    {UseR! 2019} % Event
    {Toulouse, France} % Location
    {Jul. 2012} % Date(s)
    {
      \begin{cvitems} % Description(s)
        \item {My first leading project with my friend \href{https://twitter.com/cecilesauder}{Cécile Sauder} !}
        \item {Video of the presentation available at 10:00 \href{https://www.youtube.com/watch?v=6V_nu0K_3mk&list=PL4IzsxWztPdm9_UFnWOCG7Rmw3oW5Fgju}{here}}
        \item {We performed the presentation as a play}
      \end{cvitems}
    }

%---------------------------------------------------------
\end{cventries}

%-------------------------------------------------------------------------------
%	SECTION TITLE
%-------------------------------------------------------------------------------
\cvsection{Expertise}


%-------------------------------------------------------------------------------
%	CONTENT
%-------------------------------------------------------------------------------
\begin{cventries}

%---------------------------------------------------------
  \cventry
    {Research} % Role
    {Global skills} % Title
    {} % Location
    {} % Date(s)
    {
      \begin{cvitems} % Description(s)
        \item {Expert innovator: scientific question solving with new approaches, test new tools available.}
        \item {Scientific writing: Write up experimental results for publication in peer-reviewed journals.}
        \item {Scientific drawing: Create high quality graphics and logo using vector-based design software (Illustrator/Inkscape).}
        \item {Scientific project: Prepares and writes research projects on small grant applications.}
        \item {Scientific presentation: Popularization of research (Natron / KdenLive).}
        \item {Knowledge management: Keep a bibliographic record up to date (Zotero <3).}
      \end{cvitems}
    }


%---------------------------------------------------------
  \cventry
    {Experiment} % Role
    {Biology} % Title
    {} % Location
    {} % Date(s)
    {
      \begin{cvitems} % Description(s)
        \item {Long reads sequencing: Genomic DNA by Ligation (SQK-LSK109), Rapid Sequencing (SQK-RAD004).}
        \item {Practical Immunology: Flow cytometry, Cell sorting, Mouse experimentation (lymph nodes removal, orbital blood sample), ELISA.}
        \item {Molecular Biology: DNA extraction, Transfection, Library preparation, qPCR, Western Blot, Cloning.}
        \item {Cell culture: Primary cells, HeLA cells, Bacterial culture, Viral culture.}
        \item {Experimental aquaculture: Oyster mortality monitoring, Viral particle filtration and enrichment.}
        \item {Structural biochemistry: SDS-PAGE, Chromatography, Infrared spectroscopy, NMR spectroscopy.}
      \end{cvitems}
    }

%---------------------------------------------------------
  \cventry
    {Fundamental knowledge} % Role
    {} % Title
    {} % Location
    {} % Date(s)
    {
      \begin{cvitems} % Description(s)
        \item {Immunology: Adaptive/innate immune response, T-Cells activation, Cell Ontology, Cancerology, Antiviral response, Host-pathogen interaction.}
        \item {Microbiology: Viral infection, Microbiota Dysbiosis, Holobionte, Pathogenic bacteria, Resistance, Viral evolution, Virus genetic diversity.}
        \item {Structural biochemistry: 3D crystallisation, FRET, Spectroscopy.}
      \end{cvitems}
    }

%---------------------------------------------------------
  \cventry
    {Development} % Role
    {Bioinformatics} % Title
    {} % Location
    {} % Date(s)
    {
      \begin{cvitems} % Description(s)
        \item {R: Tidyverse, Web scrapping, Data mining, Data cleaning.}
        \item {Shell scripting: Bash addict.}
        \item {NextFlow: simple analysis pipeline.}
        \item {High performance computing clusters: Manage work on HPC (PBSpro).}
        \item {Version control: Github/Gitlab.}
        \item {Reproducible analysis: Docker, Singularity, Conda, R-packages.}
      \end{cvitems}
    }

%---------------------------------------------------------
  \cventry
    {Analysis} % Role
    {} % Title
    {} % Location
    {} % Date(s)
    {
      \begin{cvitems} % Description(s)
        \item {Genome assembly: De novo or mapping assembly, Long and short reads genome assembly.}
        \item {Comparative genomics: Nucleotide variability, pan-genomics.}
        \item {Variant calling: Characterisation of SNPs and INDELs, Heterogeneity of population.}
        \item {Phylogenetics: MSA, Inference, Tree Visualisations.}
        \item {Metagenomics: Metabarcoding 16S/18S, Virus discovery, Binning contigs.}
        \item {Transcriptomics: De novo transcriptome assembly, metatranscriptomics, Differential gene expression, Gene set enrichment analysis.}
        \item {Sets analysis: sets component assembly dissection, matrix operations.}
      \end{cvitems}
    }


%---------------------------------------------------------
\end{cventries}
%-------------------------------------------------------------------------------
%	SECTION TITLE
%-------------------------------------------------------------------------------
\cvsection{Program Committees}


%-------------------------------------------------------------------------------
%	CONTENT
%-------------------------------------------------------------------------------
\begin{cvhonors}

%---------------------------------------------------------
  \cvhonor
    {Problem Writer} % Position
    {2016 CODEGATE Hacking Competition World Final} % Committee
    {S.Korea} % Location
    {2016} % Date(s)

%---------------------------------------------------------
  \cvhonor
    {Organizer \& Co-director} % Position
    {1st POSTECH Hackathon} % Committee
    {S.Korea} % Location
    {2013} % Date(s)

%---------------------------------------------------------
\end{cvhonors}

%-------------------------------------------------------------------------------
%	SECTION TITLE
%-------------------------------------------------------------------------------
\cvsection{Education}


%-------------------------------------------------------------------------------
%	CONTENT
%-------------------------------------------------------------------------------
\begin{cventries}

%---------------------------------------------------------
  \cventry
    {Master 2 in Immunology} % Degree
    {\href{https://u-paris.fr/en/universite-de-paris/}{Université de Paris}} % Institution
    {Paris, France} % Location
    {Sep. 2016 - Jun. 2017} % Date(s)
    {
      \begin{cvitems} % Description(s) bullet points
        \item {Specialization in Advanced Immunology and Immunopathologies.}
        \item {Advanced Immunology course at Pasteur institute.}
      \end{cvitems}
    }

%---------------------------------------------------------
  \cventry
    {Master 1 in  Molecular and Cell Biology} % Degree
    {\href{https://u-paris.fr/en/universite-de-paris/}{Université de Paris}} % Institution
    {Paris, France} % Location
    {Sep. 2015 - Jun. 2016} % Date(s)
    {
      \begin{cvitems} % Description(s) bullet points
        \item {Specialization in Immunology Fundamental, Immune-pathophysiology, Inflammation and Molecular pathology.}
      \end{cvitems}
    }

%---------------------------------------------------------
  \cventry
    {B.S. in Biochemistry Bioinformatics Biology} % Degree
    {\href{https://u-paris.fr/en/universite-de-paris/}{Université de Paris}} % Institution
    {Paris, France} % Location
    {Sep. 2014 - Jun. 2015} % Date(s)
    {
      \begin{cvitems} % Description(s) bullet points
        \item {Specialization in Virology and Chemistry.}
      \end{cvitems}
    }
    
%---------------------------------------------------------
  \cventry
    {University degree in technology} % Degree
    {\href{http://www.univ-tln.fr/}{University of Toulon}} % Institution
    {Toulon, France} % Location
    {Sep. 2012 - Jun. 2014} % Date(s)
    {
      \begin{cvitems} % Description(s) bullet points
        \item {Option biological analysis and biochemistry.}
      \end{cvitems}
    }

%---------------------------------------------------------
\end{cventries}

%-------------------------------------------------------------------------------
%	SECTION TITLE
%-------------------------------------------------------------------------------
\cvsection{Extracurricular Activity}


%-------------------------------------------------------------------------------
%	CONTENT
%-------------------------------------------------------------------------------
\begin{cventries}

%---------------------------------------------------------
  \cventry
    {Core Member \& President at 2013} % Affiliation/role
    {PoApper (Developers' Network of POSTECH)} % Organization/group
    {Pohang, S.Korea} % Location
    {Jun. 2010 - Jun. 2017} % Date(s)
    {
      \begin{cvitems} % Description(s) of experience/contributions/knowledge
        \item {Reformed the society focusing on software engineering and building network on and off campus.}
        \item {Proposed various marketing and network activities to raise awareness.}
      \end{cvitems}
    }

%---------------------------------------------------------
  \cventry
    {Member} % Affiliation/role
    {PLUS (Laboratory for UNIX Security in POSTECH)} % Organization/group
    {Pohang, S.Korea} % Location
    {Sep. 2010 - Oct. 2011} % Date(s)
    {
      \begin{cvitems} % Description(s) of experience/contributions/knowledge
        \item {Gained expertise in hacking \& security areas, especially about internal of operating system based on UNIX and several exploit techniques.}
        \item {Participated on several hacking competition and won a good award.}
        \item {Conducted periodic security checks on overall IT system as a member of POSTECH CERT.}
        \item {Conducted penetration testing commissioned by national agency and corporation.}
      \end{cvitems}
    }

%---------------------------------------------------------
\end{cventries}



%-------------------------------------------------------------------------------
\end{document}
