%!TEX TS-program = xelatex
%!TEX encoding = UTF-8 Unicode
% Awesome CV LaTeX Template for Cover Letter
%
% This template has been downloaded from:
% https://github.com/posquit0/Awesome-CV
%
% Authors:
% Claud D. Park <posquit0.bj@gmail.com>
% Lars Richter <mail@ayeks.de>
%
% Template license:
% CC BY-SA 4.0 (https://creativecommons.org/licenses/by-sa/4.0/)
%


%-------------------------------------------------------------------------------
% CONFIGURATIONS
%-------------------------------------------------------------------------------
% A4 paper size by default, use 'letterpaper' for US letter
\documentclass[11pt, a4paper]{awesome-cv}

% Configure page margins with geometry
\geometry{left=1.4cm, top=.8cm, right=1.4cm, bottom=1.8cm, footskip=.5cm}

% Specify the location of the included fonts
\fontdir[fonts/]

% Color for highlights
% Awesome Colors: awesome-emerald, awesome-skyblue, awesome-red, awesome-pink, awesome-orange
%                 awesome-nephritis, awesome-concrete, awesome-darknight
\colorlet{awesome}{awesome-red}
% Uncomment if you would like to specify your own color
% \definecolor{awesome}{HTML}{CA63A8}

% Colors for text
% Uncomment if you would like to specify your own color
% \definecolor{darktext}{HTML}{414141}
% \definecolor{text}{HTML}{333333}
% \definecolor{graytext}{HTML}{5D5D5D}
% \definecolor{lighttext}{HTML}{999999}

% Set false if you don't want to highlight section with awesome color
\setbool{acvSectionColorHighlight}{true}

% If you would like to change the social information separator from a pipe (|) to something else
\renewcommand{\acvHeaderSocialSep}{\quad\textbar\quad}


%-------------------------------------------------------------------------------
%	PERSONAL INFORMATION
%	Comment any of the lines below if they are not required
%-------------------------------------------------------------------------------
% Available options: circle|rectangle,edge/noedge,left/right
\photo[circle,noedge,left]{./profile.png}
\name{Jean}{Delmotte}
\position{Bioinformaticien{\enskip\cdotp\enskip}Computational Biologist{\enskip\cdotp\enskip}Immunologist}
\address{26, rue de la Verrerie, Montpellier, 34000, France}

\mobile{(+33) 06-8721-5961}
\email{jeandelmotte0392@gmail.com}
\homepage{www.jdlab.fr/}
\github{propan2one}
%\linkedin{jean-delmotte}
% \gitlab{gitlab-id}
% \stackoverflow{SO-id}{SO-name}
 \twitter{@DrStagiaire}
% \skype{skype-id}
% \reddit{reddit-id}
% \medium{madium-id}
% \googlescholar{googlescholar-id}{name-to-display}
%% \firstname and \lastname will be used
% \googlescholar{googlescholar-id}{}
% \extrainfo{extra informations}

\quote{``Talk is cheap. Show me the code."}


%-------------------------------------------------------------------------------
%	LETTER INFORMATION
%	All of the below lines must be filled out
%-------------------------------------------------------------------------------
% The company being applied to
\recipient
  {Stem Genomics}
  {IRMB - Hopital Saint Eloi, 80 avenue Augustin Fliche, Montpellier}
% The date on the letter, default is the date of compilation
\letterdate{\today}
% The title of the letter
\lettertitle{Candidature à un poste d'ingénieur en bioinformatique}
% \lettertitle{Job Application for Bioinformatic engineer in functional and evolutionary genomics}
% How the letter is opened
\letteropening{Dr.Chapal,}
% \letteropening{Dear Mr./Ms./Dr. LastName,}
% How the letter is closed
\letterclosing{Sincerely,}
% Any enclosures with the letter
\letterenclosure[Attached]{Curriculum Vitae}


%-------------------------------------------------------------------------------
\begin{document}

% Print the header with above personal informations
% Give optional argument to change alignment(C: center, L: left, R: right)
\makecvheader[R]

% Print the footer with 3 arguments(<left>, <center>, <right>)
% Leave any of these blank if they are not needed
\makecvfooter
  {\today}
  {Jean Delmotte~~~·~~~Cover Letter}
  {\thepage}
% Print the title with above letter informations
\makelettertitle

%-------------------------------------------------------------------------------
%	LETTER CONTENT
%-------------------------------------------------------------------------------
\begin{cvletter}

Après plusieurs années dans la recherche académique à l’étude des interactions hôtes et pathogènes, je souhaite maintenant intégrer une équipe de RD sur des sujets à l’interface entre immunologie et bio-informatique sur des sujets plus ambitieux.

\lettersection{Pourquoi travailler à Stem Genomics}

Les thérapies cellulaires personnalisées sont amenées à se développer en cancérologie et en médecine régénérative dans les prochaines années et donc implicitement l'analyse et le contrôle de l'intégrité cellulaire dans ces traitements sont eux aussi amenés à augmenter. De plus, il semble que votre travail valorise le domaine académique, comme le montre les publications d'articles scientifiques, en parallèle de vos activités d'entreprise.

\lettersection{A propos de moi}
Depuis maintenant plus de six ans, mes recherches portent sur la compréhension des interactions hôtes pathogènes. J'ai été « infecté » par la passion de la recherche fondamentale à la suite de mon travail sur l'étude du VIH dans un laboratoire de l'hôpital Saint Louis. Après une formation d'immunologie, à Paris, je suis actuellement en train de finaliser mon travail de thèse en bio-informatique. J'essaie maintenant de construire une vision de la recherche plus proche de mes principes et idéaux. Mes points forts les plus remarquables sont la curiosité et l'inventivité, ce qui s'avère particulièrement utile dans mon travail où l'on peut sans cesse découvrir de nouvelles technologies et aller plus en profondeur dans les analyses.

\lettersection{Pourquoi moi ?}
Lors de mes études d'immunologie, j'ai rapidement compris l'intérêt des thérapies cellulaires au travers des applications sur les CAR-T cells. Mes trois ans de thèse m'ont permis d'ajouter de nombreuses compétences supplémentaires tant théoriques, comme une meilleure compréhension de la biologie évolutive, que pratique telle que le développement de pipeline de bio-informatique. J'ai ainsi une vision plus moderne des pressions de sélection entrainant des changements génomiques dans différents organismes et comment les identifier. C'est pourquoi je pense être à même de vous aider sur les analyses de contrôle de l'intégrité génétique des cellules-souches.


\end{cvletter}

%-------------------------------------------------------------------------------
% Preamble
% Main text
%\includegraphics[scale=0.08]{signature.jpg}

% Print the signature and enclosures with above letter informations
\makeletterclosing


\end{document}
