%!TEX TS-program = xelatex
%!TEX encoding = UTF-8 Unicode
% Awesome CV LaTeX Template for Cover Letter
%
% This template has been downloaded from:
% https://github.com/posquit0/Awesome-CV
%
% Authors:
% Claud D. Park <posquit0.bj@gmail.com>
% Lars Richter <mail@ayeks.de>
%
% Template license:
% CC BY-SA 4.0 (https://creativecommons.org/licenses/by-sa/4.0/)
%


%-------------------------------------------------------------------------------
% CONFIGURATIONS
%-------------------------------------------------------------------------------
% A4 paper size by default, use 'letterpaper' for US letter
\documentclass[11pt, a4paper]{awesome-cv}

% Configure page margins with geometry
\geometry{left=1.4cm, top=.8cm, right=1.4cm, bottom=1.8cm, footskip=.5cm}

% Specify the location of the included fonts
\fontdir[fonts/]

% Color for highlights
% Awesome Colors: awesome-emerald, awesome-skyblue, awesome-red, awesome-pink, awesome-orange
%                 awesome-nephritis, awesome-concrete, awesome-darknight
%\colorlet{awesome}{awesome-skyblue}
% Uncomment if you would like to specify your own color
\definecolor{awesome}{HTML}{800080}

% Colors for text
% Uncomment if you would like to specify your own color
% \definecolor{darktext}{HTML}{414141}
% \definecolor{text}{HTML}{333333}
% \definecolor{graytext}{HTML}{5D5D5D}
% \definecolor{lighttext}{HTML}{999999}

% Set false if you don't want to highlight section with awesome color
\setbool{acvSectionColorHighlight}{true}

% If you would like to change the social information separator from a pipe (|) to something else
\renewcommand{\acvHeaderSocialSep}{\quad\textbar\quad}


%-------------------------------------------------------------------------------
%	PERSONAL INFORMATION
%	Comment any of the lines below if they are not required
%-------------------------------------------------------------------------------
% Available options: circle|rectangle,edge/noedge,left/right
\photo[circle,noedge,left]{./profile.png}
\name{Jean}{Delmotte}
\position{Bioinformatician{\enskip\cdotp\enskip}Computational Biologist{\enskip\cdotp\enskip}Immunologist}
\address{285, rue André Philip, Lyon, 69003, France}

\mobile{(+33) 06-8721-5961}
\email{jeandelmotte0392@gmail.com}
\homepage{www.jdlab.fr/}
\github{propan2one}
%\linkedin{jean-delmotte}
% \gitlab{gitlab-id}
% \stackoverflow{SO-id}{SO-name}
 \twitter{@DrStagiaire}
% \skype{skype-id}
% \reddit{reddit-id}
% \medium{madium-id}
% \googlescholar{googlescholar-id}{name-to-display}
%% \firstname and \lastname will be used
% \googlescholar{googlescholar-id}{}
% \extrainfo{extra informations}

\quote{``Talk is cheap. Show me the code."}


%-------------------------------------------------------------------------------
%	LETTER INFORMATION
%	All of the below lines must be filled out
%-------------------------------------------------------------------------------
% The company being applied to
\recipient
  {}
  {}
% The date on the letter, default is the date of compilation
\letterdate{\today}
% The title of the letter
\lettertitle{Candidature au poste de scientifique bio-computationnelle}
% \lettertitle{Job Application for Bioinformatic engineer in functional and evolutionary genomics}
% How the letter is opened
\letteropening{Monsieur Vernay,}
% How the letter is closed
\letterclosing{Sincerely,}
% Any enclosures with the letter
\letterenclosure[Attached]{Curriculum Vitae}


%-------------------------------------------------------------------------------
\begin{document}

% Print the header with above personal informations
% Give optional argument to change alignment(C: center, L: left, R: right)
\makecvheader[L]

% Print the footer with 3 arguments(<left>, <center>, <right>)
% Leave any of these blank if they are not needed
\makecvfooter
  {\today}
  {Jean Delmotte~~~·~~~Cover Letter}
  {\thepage}
% Print the title with above letter informations
\makelettertitle

%-------------------------------------------------------------------------------
%	LETTER CONTENT
%-------------------------------------------------------------------------------
\begin{cvletter}

Par la présente lettre et les CVs qui l’accompagne, j’ai le plaisir de vous soumettre ma candidature pour le poste de scientifique biocomputationnelle au sein de Sanofi.

%L’étude des sciences du vivant m’a progressivement amené à utiliser de manière intensive des méthodes d’analyse en bioinformatique. Ces recherches m’ont permis de questionner l’importante diversité du monde vivant en biologie. Je souhaite maintenant apporter mon expertise dans une équipe de recherche orientée sur des projets à l’interface entre la recherche fondamentale et l’application clinique.

\lettersection{Qui suis-je}
Je suis à la fois un bioinformaticien et un biologiste. Cette double compétence me permet d'être un interlocuteur privilégié pour faire communiquer ces deux mondes. Mes connaissances et mes compétences peuvent vous aider à 
traiter d'importante quantité de données inhérente au développement vaccinale et leurs intégrations dans l'environnement digitale. De plus, j'aime transmettre mes connaissances à mes collaborateurs leurs permettant ainsi de comprendre comment fonctionne une analyse afin d'éviter le syndrome de la boite noire. Enfin, j'ai une approche contemporaine de la science, ce qui, couplé à une grande motivation me permet d’être un important moteur d’innovation.

%m'engage également en créer des liens entre les différ.

%Depuis maintenant plus de dix ans, je suis passionné de microbiologie, virologie et immunologie. Il y a quatre ans, j’ai découvert le potentiel des approches de séquençage à haut débit dans l’étude de phénomènes biologiques complexes. C’est dans l’optique de monter en compétence dans ce domaine que j’ai fait le choix de m’orienter sur une thèse présentant des approches expérimentales et analytiques de type “omique” (génomique, transcriptomique, épigénomique, protéomique). J’ai réussi à me spécialiser dans l’analyse de la diversité génomique où j’essaye d’apporter la composante pangénomique. J'ai également exploré d'autres aspects fondamentaux de la recherche, comme la reproductibilité des résultats, leurs partages et leurs valorisations. Des valeurs qui me correspondent et qui sont très ancrées dans le développement logiciel et la bioinformatique.

\lettersection{Pourquoi moi}

J'attache un engagement important à collaborer auprès des différent groupes auquel j'amène mon support dans des des domaines de recherche très variés : génomique, biologie moléculaire, biologie structural, entre autres.

Grâce à mes travaux académiques sur l’analyse de données biologiques, j’ai eu l’occasion d’être associé à l'ensemble des étapes clés dans le développement de nombreux projets scientifiques. À l’aide de ma double compétence de biologie expérimental et de bioinformatique, je suis un interlocuteur privilégié pour faire communiquer ces deux mondes. J'ai une approche contemporaine de la recherche, ce qui, couplé à une grande motivation me permet d’être un important moteur d’innovation. C'est pourquoi je pense être à même de vous apporter un support d’analyse bioinformatique de qualité pour vos différents projets scientifiques.

\end{cvletter}

%-------------------------------------------------------------------------------
% Preamble
% Main text
%\includegraphics[scale=0.08]{signature.jpg}

% Print the signature and enclosures with above letter informations
\makeletterclosing


\end{document}
