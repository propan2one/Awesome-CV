%!TEX TS-program = xelatex
%!TEX encoding = UTF-8 Unicode
% Awesome CV LaTeX Template for Cover Letter
%
% This template has been downloaded from:
% https://github.com/posquit0/Awesome-CV
%
% Authors:
% Claud D. Park <posquit0.bj@gmail.com>
% Lars Richter <mail@ayeks.de>
%
% Template license:
% CC BY-SA 4.0 (https://creativecommons.org/licenses/by-sa/4.0/)
%


%-------------------------------------------------------------------------------
% CONFIGURATIONS
%-------------------------------------------------------------------------------
% A4 paper size by default, use 'letterpaper' for US letter
\documentclass[11pt, a4paper]{awesome-cv}

% Configure page margins with geometry
\geometry{left=1.4cm, top=.8cm, right=1.4cm, bottom=1.8cm, footskip=.5cm}

% Specify the location of the included fonts
\fontdir[fonts/]

% Color for highlights
% Awesome Colors: awesome-emerald, awesome-skyblue, awesome-red, awesome-pink, awesome-orange
%                 awesome-nephritis, awesome-concrete, awesome-darknight
\colorlet{awesome}{awesome-skyblue}
% Uncomment if you would like to specify your own color
% \definecolor{awesome}{HTML}{CA63A8}

% Colors for text
% Uncomment if you would like to specify your own color
% \definecolor{darktext}{HTML}{414141}
% \definecolor{text}{HTML}{333333}
% \definecolor{graytext}{HTML}{5D5D5D}
% \definecolor{lighttext}{HTML}{999999}

% Set false if you don't want to highlight section with awesome color
\setbool{acvSectionColorHighlight}{true}

% If you would like to change the social information separator from a pipe (|) to something else
\renewcommand{\acvHeaderSocialSep}{\quad\textbar\quad}


%-------------------------------------------------------------------------------
%	PERSONAL INFORMATION
%	Comment any of the lines below if they are not required
%-------------------------------------------------------------------------------
% Available options: circle|rectangle,edge/noedge,left/right
\photo[circle,noedge,left]{./profile.png}
\name{Jean}{Delmotte}
\position{Bioinformatician{\enskip\cdotp\enskip}Computational Biologist{\enskip\cdotp\enskip}Immunologist}
\address{26, rue de la Verrerie, Montpellier, 34000, France}

\mobile{(+33) 06-8721-5961}
\email{jeandelmotte0392@gmail.com}
\homepage{www.jdlab.fr/}
\github{propan2one}
%\linkedin{jean-delmotte}
% \gitlab{gitlab-id}
% \stackoverflow{SO-id}{SO-name}
 \twitter{@DrStagiaire}
% \skype{skype-id}
% \reddit{reddit-id}
% \medium{madium-id}
% \googlescholar{googlescholar-id}{name-to-display}
%% \firstname and \lastname will be used
% \googlescholar{googlescholar-id}{}
% \extrainfo{extra informations}

\quote{``Talk is cheap. Show me the code."}


%-------------------------------------------------------------------------------
%	LETTER INFORMATION
%	All of the below lines must be filled out
%-------------------------------------------------------------------------------
% The company being applied to
\recipient
  {Centre International de Recherche en Infectiologie}
  {Équipe Pathogénie des infections à staphylocoques, Lyon (France)}
% The date on the letter, default is the date of compilation
\letterdate{\today}
% The title of the letter
\lettertitle{Candidature à un poste d’ingénieur en bioinformatique}
% \lettertitle{Job Application for Bioinformatic engineer in functional and evolutionary genomics}
% How the letter is opened
\letteropening{M. Medina,}
% \letteropening{Dear Mr./Ms./Dr. LastName,}
% How the letter is closed
\letterclosing{Sincerely,}
% Any enclosures with the letter
\letterenclosure[Attached]{Curriculum Vitae}


%-------------------------------------------------------------------------------
\begin{document}

% Print the header with above personal informations
% Give optional argument to change alignment(C: center, L: left, R: right)
\makecvheader[R]

% Print the footer with 3 arguments(<left>, <center>, <right>)
% Leave any of these blank if they are not needed
\makecvfooter
  {\today}
  {Jean Delmotte~~~·~~~Cover Letter}
  {\thepage}
% Print the title with above letter informations
\makelettertitle

%-------------------------------------------------------------------------------
%	LETTER CONTENT
%-------------------------------------------------------------------------------
\begin{cvletter}

L’étude des sciences du vivant m’a progressivement amené à utiliser de manière intensive des méthodes d’analyse en bioinformatique. Ces recherches m’ont permis de questionner l’importante diversité du monde vivant en biologie. Je souhaite maintenant apporter mon expertise dans une équipe de recherche orientée sur des projets à l’interface entre la recherche fondamentale et l’application clinique.

\lettersection{Pourquoi travailler avec vous}

L’apparition de bactéries résistantes aux antibiotiques est l’un des problèmes de santé publique majeur à venir. L’utilisation de bactériophages thérapeutiques semble prometteur pour combattre ces bactéries multirésistantes. Au sein de votre service des maladies infectieuses de l’hôpital de La Croix-Rousse vous avez récemment réussi à guérir des infections ostéoarticulaires graves à Staphylocoque démontrant ainsi l'efficacité et la faisabilité de la phagothérapie chez des patients. En parallèle, au sein de l’équipe ”Pathogénie des infections à staphylocoques” du CIRI, des recherches utilisant des outils de bioinformatique ont amélioré la compréhension de la pathologie induite par des staphylocoques. Ces axes de recherches s’inscrivent dans une vision moderne de l’étude des maladies infectieuses que je partage également.

\lettersection{A propos de moi}

Depuis maintenant plus de dix ans, je suis passionné de microbiologie, virologie et immunologie. Il y a quatre ans, j’ai découvert le potentiel des approches de séquençage à haut débit dans l’étude de phénomènes biologiques complexes. C’est dans l’optique de monter en compétence dans ce domaine que j’ai fait le choix de m’orienter sur une thèse avec une forte composante d’analyse en génomique. J’ai réussi à me spécialiser dans l’analyse de la diversité génomique où j’essaye d’apporter la composante pangénomique. J'ai également exploré d'autres aspects fondamentaux de la recherche, comme la reproductibilité des résultats, leurs partages et leurs valorisations. Des valeurs qui me correspondent et qui sont très ancrées dans le développement logiciel et la bioinformatique.

\lettersection{Pourquoi moi}
Grâce à mes travaux académiques sur l’analyse de données biologiques, j’ai eu l’occasion d’être associé à l'ensemble des étapes clés dans le développement de nombreux projets scientifiques. À l’aide de ma double compétence de biologie expérimental et de bioinformatique, je suis un interlocuteur privilégié pour faire communiquer ces deux mondes. J'ai une approche contemporaine de la recherche, ce qui, couplé à une grande motivation me permet d’être un important moteur d’innovation. C'est pourquoi je pense être à même de vous aider dans l’étude des mécanismes physiopathologiques des infections à staphylocoques et dans l’identification et caractérisation de phages que vous étudiez. 

\end{cvletter}

%-------------------------------------------------------------------------------
% Preamble
% Main text
%\includegraphics[scale=0.08]{signature.jpg}

% Print the signature and enclosures with above letter informations
\makeletterclosing


\end{document}
