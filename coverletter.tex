%!TEX TS-program = xelatex
%!TEX encoding = UTF-8 Unicode
% Awesome CV LaTeX Template for Cover Letter
%
% This template has been downloaded from:
% https://github.com/posquit0/Awesome-CV
%
% Authors:
% Claud D. Park <posquit0.bj@gmail.com>
% Lars Richter <mail@ayeks.de>
%
% Template license:
% CC BY-SA 4.0 (https://creativecommons.org/licenses/by-sa/4.0/)
%


%-------------------------------------------------------------------------------
% CONFIGURATIONS
%-------------------------------------------------------------------------------
% A4 paper size by default, use 'letterpaper' for US letter
\documentclass[11pt, a4paper]{awesome-cv}

% Configure page margins with geometry
\geometry{left=1.4cm, top=.8cm, right=1.4cm, bottom=1.8cm, footskip=.5cm}

% Specify the location of the included fonts
\fontdir[fonts/]

% Color for highlights
% Awesome Colors: awesome-emerald, awesome-skyblue, awesome-red, awesome-pink, awesome-orange
%                 awesome-nephritis, awesome-concrete, awesome-darknight
\colorlet{awesome}{awesome-red}
% Uncomment if you would like to specify your own color
% \definecolor{awesome}{HTML}{CA63A8}

% Colors for text
% Uncomment if you would like to specify your own color
% \definecolor{darktext}{HTML}{414141}
% \definecolor{text}{HTML}{333333}
% \definecolor{graytext}{HTML}{5D5D5D}
% \definecolor{lighttext}{HTML}{999999}

% Set false if you don't want to highlight section with awesome color
\setbool{acvSectionColorHighlight}{true}

% If you would like to change the social information separator from a pipe (|) to something else
\renewcommand{\acvHeaderSocialSep}{\quad\textbar\quad}


%-------------------------------------------------------------------------------
%	PERSONAL INFORMATION
%	Comment any of the lines below if they are not required
%-------------------------------------------------------------------------------
% Available options: circle|rectangle,edge/noedge,left/right
\photo[circle,noedge,left]{./profile.png}
\name{Jean}{Delmotte}
\position{Bioinformaticien{\enskip\cdotp\enskip}Computational Biologist{\enskip\cdotp\enskip}Immunologist}
\address{26, rue de la Verrerie, Montpellier, 34000, France}

\mobile{(+33) 06-8721-5961}
\email{jeandelmotte0392@gmail.com}
\homepage{www.jdlab.fr/}
\github{propan2one}
\linkedin{jean-delmotte}
% \gitlab{gitlab-id}
% \stackoverflow{SO-id}{SO-name}
 \twitter{@DrStagiaire}
% \skype{skype-id}
% \reddit{reddit-id}
% \medium{madium-id}
% \googlescholar{googlescholar-id}{name-to-display}
%% \firstname and \lastname will be used
% \googlescholar{googlescholar-id}{}
% \extrainfo{extra informations}

\quote{``Talk is cheap. Show me the code."}


%-------------------------------------------------------------------------------
%	LETTER INFORMATION
%	All of the below lines must be filled out
%-------------------------------------------------------------------------------
% The company being applied to
\recipient
  {L’Institut des Sciences de l’Evolution de Montpellier (ISEM)}
  {Equipe : Evolution, Vecteurs, Adaptation et Symbioses}
% The date on the letter, default is the date of compilation
\letterdate{\today}
% The title of the letter
\lettertitle{Candidature à un poste d'ingénieur d’études en bioinformatique}
% \lettertitle{Job Application for Bioinformatic engineer in functional and evolutionary genomics}
% How the letter is opened
\letteropening{Dr.Weill,}
% \letteropening{Dear Mr./Ms./Dr. LastName,}
% How the letter is closed
\letterclosing{Sincerely,}
% Any enclosures with the letter
\letterenclosure[Attached]{Curriculum Vitae}


%-------------------------------------------------------------------------------
\begin{document}

% Print the header with above personal informations
% Give optional argument to change alignment(C: center, L: left, R: right)
\makecvheader[R]

% Print the footer with 3 arguments(<left>, <center>, <right>)
% Leave any of these blank if they are not needed
\makecvfooter
  {\today}
  {Jean Delmotte~~~·~~~Cover Letter}
  {\thepage}
% Print the title with above letter informations
\makelettertitle

%-------------------------------------------------------------------------------
%	LETTER CONTENT
%-------------------------------------------------------------------------------
\begin{cvletter}

Après plusieurs années dans la recherche sur les interactions hôtes et pathogènes entre des huîtres et des virus, je souhaite maintenant m’investir dans l’étude de nouveaux modèles, au sein d’un laboratoire intéressé par des questions de biologie évolutives.


\lettersection{Pourquoi travailler à l’ISEM}

Votre laboratoire poursuit actuellement des recherches sur une meilleure compréhension des interactions entre les bactéries endosymbiotiques des moustiques avec notamment des hypothèses évolutives sur les pressions sélectives amenant une diversification génétique. Travailler sur ce genre de questions m’intéresse et me permettra d’aller toujours plus loin dans mes méthodologies d’analyses. De plus, vos connaissances en biologie évolutive m’aideront dans mes propres recherches. Enfin, conduire ces investigations me semble crucial à notre époque pour développer des méthodes de lutte contre les moustiques.


\lettersection{A propos de moi}
Depuis maintenant plus de 6 ans, mes recherches portent sur la compréhension des interactions hôtes pathogènes. J'ai été 'infecté' par la passion de la recherche fondamentale à la suite de mon travail sur l'étude du VIH dans un laboratoire de l’hôpital Saint Louis. Après une formation d’immunologiste, je suis actuellement en train de finaliser mon travail de thèse en bio-informatique que je défendrai début mars 2021. J’essaie maintenant de construire une vision de la recherche plus proche de mes principes et idéaux : une recherche orientée vers la science ouverte, visant à en faciliter le libre accès. Mes points forts les plus remarquables sont la curiosité et l’inventivité, ce qui s’avère particulièrement utile dans mon travail où l’on peut sans cesse découvrir de nouvelles technologies et aller plus en profondeur dans les analyses.

\lettersection{Pourquoi moi ?}
J’ai été amené à développer de nombreux pipelines d’analyse qui sont en l’état actuel des connaissances des outils de bio-informatiques. C’est pourquoi je pense être à même de vous aider sur les analyses de metaviromiques, sur les assemblages de génomes et la pan-génomique bactérienne avec les gènes des bactéries endosymbiotiques. De plus, je suis en train de mener un projet de vulgarisation scientifique, où j’aimerais mettre en avant la production scientifique et je pense que cela serait intéressant avec un sujet comme le vôtre. Pour finir, mon doctorat pourrait être un frein à ce poste, toutefois je ne le validerai qu’en mars, je pense être éligible à un poste d’ingénieur d'études.


\end{cvletter}


%-------------------------------------------------------------------------------
% Print the signature and enclosures with above letter informations
\makeletterclosing

\end{document}
