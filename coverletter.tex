%!TEX TS-program = xelatex
%!TEX encoding = UTF-8 Unicode
% Awesome CV LaTeX Template for Cover Letter
%
% This template has been downloaded from:
% https://github.com/posquit0/Awesome-CV
%
% Authors:
% Claud D. Park <posquit0.bj@gmail.com>
% Lars Richter <mail@ayeks.de>
%
% Template license:
% CC BY-SA 4.0 (https://creativecommons.org/licenses/by-sa/4.0/)
%


%-------------------------------------------------------------------------------
% CONFIGURATIONS
%-------------------------------------------------------------------------------
% A4 paper size by default, use 'letterpaper' for US letter
\documentclass[11pt, a4paper]{awesome-cv}

% Configure page margins with geometry
\geometry{left=1.4cm, top=.8cm, right=1.4cm, bottom=1.8cm, footskip=.5cm}

% Specify the location of the included fonts
\fontdir[fonts/]

% Color for highlights
% Awesome Colors: awesome-emerald, awesome-skyblue, awesome-red, awesome-pink, awesome-orange
%                 awesome-nephritis, awesome-concrete, awesome-darknight
\colorlet{awesome}{awesome-red}
% Uncomment if you would like to specify your own color
% \definecolor{awesome}{HTML}{CA63A8}

% Colors for text
% Uncomment if you would like to specify your own color
% \definecolor{darktext}{HTML}{414141}
% \definecolor{text}{HTML}{333333}
% \definecolor{graytext}{HTML}{5D5D5D}
% \definecolor{lighttext}{HTML}{999999}

% Set false if you don't want to highlight section with awesome color
\setbool{acvSectionColorHighlight}{true}

% If you would like to change the social information separator from a pipe (|) to something else
\renewcommand{\acvHeaderSocialSep}{\quad\textbar\quad}


%-------------------------------------------------------------------------------
%	PERSONAL INFORMATION
%	Comment any of the lines below if they are not required
%-------------------------------------------------------------------------------
% Available options: circle|rectangle,edge/noedge,left/right
\photo[circle,noedge,left]{./profile.png}
\name{Jean}{Delmotte}
\position{Computational Biologist{\enskip\cdotp\enskip}Bioinformaticien}
\address{26, rue de la Verrerie, Montpellier, 34000, France}

\mobile{(+33) 06-8721-5961}
\email{jeandelmotte0392@gmail.com}
\homepage{www.jdlab.fr/}
\github{propan2one}
\linkedin{jean-delmotte}
% \gitlab{gitlab-id}
% \stackoverflow{SO-id}{SO-name}
 \twitter{@DrStagiaire}
% \skype{skype-id}
% \reddit{reddit-id}
% \medium{madium-id}
% \googlescholar{googlescholar-id}{name-to-display}
%% \firstname and \lastname will be used
% \googlescholar{googlescholar-id}{}
% \extrainfo{extra informations}

\quote{``Talk is cheap. Show me the code."}


%-------------------------------------------------------------------------------
%	LETTER INFORMATION
%	All of the below lines must be filled out
%-------------------------------------------------------------------------------
% The company being applied to
\recipient
  {Virostyle team}
  {MIVEGEC - UMR5290\\MONTPELLIER}
% The date on the letter, default is the date of compilation
\letterdate{\today}
% The title of the letter
\lettertitle{Job Application for Bioinformatic engineer in functional and evolutionary genomics}
% \lettertitle{Job Application for Bioinformatic Engineer}
% How the letter is opened
\letteropening{Dear Dr.Bravo,}
% \letteropening{Dear Mr./Ms./Dr. LastName,}
% How the letter is closed
\letterclosing{Sincerely,}
% Any enclosures with the letter
\letterenclosure[Attached]{Curriculum Vitae}


%-------------------------------------------------------------------------------
\begin{document}

% Print the header with above personal informations
% Give optional argument to change alignment(C: center, L: left, R: right)
\makecvheader[R]

% Print the footer with 3 arguments(<left>, <center>, <right>)
% Leave any of these blank if they are not needed
\makecvfooter
  {\today}
  {Jean Delmotte~~~·~~~Cover Letter}
  {\thepage}
% Print the title with above letter informations
\makelettertitle

%-------------------------------------------------------------------------------
%	LETTER CONTENT
%-------------------------------------------------------------------------------
\begin{cvletter}

\lettersection{About Me}

I am a PhD Student fellow in the Host-Pathogen-Environment Interactions (IHPE) research group at the University of Montpellier (UM). For more than 6 years now, my research has focused on understanding host-pathogen interactions. I was "infected" with a passion for basic research as a result of my work on the study of HIV when I was younger. I am now in the process of finalizing my thesis work which I will defend in December of this year. With this 3 years work, I was able to understand how academic research, and myself, works from a professional point of view. I know my greatest strengths: I have a great capacity for innovation, I am able to project myself quite far in the analyzes and I tend to be professional as much as possible. This is how I ended up doing bioinformatics in 3 years, training myself from zero until reaching the best international standard. I also know my weaknesses, I don't accept injustice, I am particularly intransigent (especially about data analysis) and I don't easily accept dogmas. Which explains why my thesis did not always go well. I really like working in science, but everything I do, I do 200 percents, because I don't just work with my ideas, I also work with my principles and ideals like open science, knowledge sharing and collaborations. Outside of work I am a very open-minded person, I like to debate or simply know people's opinions, especially over a beer!

\lettersection{Why Virostyle team from MIVEGEC}
I attended one of your talks on ecology and evolution of oncogenic papillomaviruses at the 2019 Microbioccitanie congress and I really appreciated it. My thesis laboratory is neither specialized in virology nor in evolutionary study. With you I could finally address fundamental questions that interest me on host and pathogen relationships such as the adaptation mechanisms of the viral genome to its host. With your expertise and my innovations, I think we can produce good science.

\lettersection{Why Me?}
I saw that your team is interested in the evolution of polyomavirus viral populations. This is exactly the subject that I dealt with during my thesis, but on another virus: OsHV-1 which infects oysters. Our working hypothesis was the same as yours, with the involvement of viral populations on the disease. I therefore characterized the viral populations by developing analyzes such as comparative genomics and variant calls, the objective being to do phylodynamics. The paper on this research is almost finished and presents many improvements including the latest tools available to date. All the concepts and results produced are the fruit of my research. All the pipelines that will be useful to you are already developed and those in a repeatable fashion. Recently, I have been working in close collaboration with the people of the bioinformatics department of ifremer in order to transform them into a NextFlow pipeline. Finally, the latest technical advances have led me to develop a tool that allows characterizing all of the viral genetic diversity for a given specie, but not in a linear fashion. This tool has an innovative approach to synchronize comparative genomics and variant calling analyzes and it looks like pan-genomics analyzes.

\end{cvletter}


%-------------------------------------------------------------------------------
% Print the signature and enclosures with above letter informations
\makeletterclosing

\end{document}
