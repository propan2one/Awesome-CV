%!TEX TS-program = xelatex
%!TEX encoding = UTF-8 Unicode
% Awesome CV LaTeX Template for Cover Letter
%
% This template has been downloaded from:
% https://github.com/posquit0/Awesome-CV
%
% Authors:
% Claud D. Park <posquit0.bj@gmail.com>
% Lars Richter <mail@ayeks.de>
%
% Template license:
% CC BY-SA 4.0 (https://creativecommons.org/licenses/by-sa/4.0/)
%


%-------------------------------------------------------------------------------
% CONFIGURATIONS
%-------------------------------------------------------------------------------
% A4 paper size by default, use 'letterpaper' for US letter
\documentclass[11pt, a4paper]{awesome-cv}

% Configure page margins with geometry
\geometry{left=1.4cm, top=.8cm, right=1.4cm, bottom=1.8cm, footskip=.5cm}

% Specify the location of the included fonts
\fontdir[fonts/]

% Color for highlights
% Awesome Colors: awesome-emerald, awesome-skyblue, awesome-red, awesome-pink, awesome-orange
%                 awesome-nephritis, awesome-concrete, awesome-darknight
\colorlet{awesome}{awesome-red}
% Uncomment if you would like to specify your own color
% \definecolor{awesome}{HTML}{CA63A8}

% Colors for text
% Uncomment if you would like to specify your own color
% \definecolor{darktext}{HTML}{414141}
% \definecolor{text}{HTML}{333333}
% \definecolor{graytext}{HTML}{5D5D5D}
% \definecolor{lighttext}{HTML}{999999}

% Set false if you don't want to highlight section with awesome color
\setbool{acvSectionColorHighlight}{true}

% If you would like to change the social information separator from a pipe (|) to something else
\renewcommand{\acvHeaderSocialSep}{\quad\textbar\quad}


%-------------------------------------------------------------------------------
%	PERSONAL INFORMATION
%	Comment any of the lines below if they are not required
%-------------------------------------------------------------------------------
% Available options: circle|rectangle,edge/noedge,left/right
\photo[circle,noedge,left]{./profile.png}
\name{Jean}{Delmotte}
\position{Bioinformaticien{\enskip\cdotp\enskip}Computational Biologist{\enskip\cdotp\enskip}Immunologist}
\address{26, rue de la Verrerie, Montpellier, 34000, France}

\mobile{(+33) 06-8721-5961}
\email{jeandelmotte0392@gmail.com}
\homepage{www.jdlab.fr/}
\github{propan2one}
%\linkedin{jean-delmotte}
% \gitlab{gitlab-id}
% \stackoverflow{SO-id}{SO-name}
 \twitter{@DrStagiaire}
% \skype{skype-id}
% \reddit{reddit-id}
% \medium{madium-id}
% \googlescholar{googlescholar-id}{name-to-display}
%% \firstname and \lastname will be used
% \googlescholar{googlescholar-id}{}
% \extrainfo{extra informations}

\quote{``Talk is cheap. Show me the code."}


%-------------------------------------------------------------------------------
%	LETTER INFORMATION
%	All of the below lines must be filled out
%-------------------------------------------------------------------------------
% The company being applied to
\recipient
  {Soladis}
  {6-8 rue Bellecombe, 69006 Lyon (France)}
% The date on the letter, default is the date of compilation
\letterdate{\today}
% The title of the letter
\lettertitle{Candidature à un poste d'ingénieur en bioinformatique}
% \lettertitle{Job Application for Bioinformatic engineer in functional and evolutionary genomics}
% How the letter is opened
\letteropening{Madame, Monsieur}
% \letteropening{Dear Mr./Ms./Dr. LastName,}
% How the letter is closed
\letterclosing{Sincerely,}
% Any enclosures with the letter
\letterenclosure[Attached]{Curriculum Vitae}


%-------------------------------------------------------------------------------
\begin{document}

% Print the header with above personal informations
% Give optional argument to change alignment(C: center, L: left, R: right)
\makecvheader[R]

% Print the footer with 3 arguments(<left>, <center>, <right>)
% Leave any of these blank if they are not needed
\makecvfooter
  {\today}
  {Jean Delmotte~~~·~~~Cover Letter}
  {\thepage}
% Print the title with above letter informations
\makelettertitle

%-------------------------------------------------------------------------------
%	LETTER CONTENT
%-------------------------------------------------------------------------------
\begin{cvletter}

Après plusieurs années dans la recherche académique et plus particulièrement l’étude des interactions hôtes et pathogènes, je souhaite maintenant intégrer une équipe spécialisée sur l’analyse de données de type omiques.

\lettersection{Pourquoi travailler à Soladis ?}
De nos jours, la recherche sur les technologies omiques est au cœur de nombreux défis scientifiques tels que l’étude des maladies infectieuses ou la lutte contre le cancer. Ces recherches sont amenées à se développer dans les prochaines années et la société Soladis bénéficie d’une expertise importante dans ce domaine. De plus, ses liens avec la recherche académique, et d’autres partenaires industriels en font un acteur majeur sur les projets ambitieux à venir.

\lettersection{A propos de moi}
Depuis maintenant plus de six ans, mes recherches portent sur la compréhension des interactions hôtes pathogènes. J'ai été « infecté » par la passion de la recherche et de l'innovation à la suite d’un premier travail sur l'étude du VIH. Suite à une formation d'immunologie à l’Institut Pasteur, j’ai découvert le potentiel du traitement de données par bioinformatique. J’ai depuis considérablement amélioré mes compétences sur le sujet en investissant une partie de mon temps pour maîtriser les outils de développement informatique. Mes points forts sont la curiosité et l'inventivité, ce qui s'avère particulièrement utile dans mon travail où l'on peut sans cesse découvrir de nouvelles technologies et aller plus en profondeur dans les analyses.

\lettersection{Pourquoi moi ?}
Grâce à mes formations en biologie comme la biochimie, la microbiologie et l’immunologie, mais aussi mes travaux de thèse centrés sur l’évaluation et l’analyse de données omiques, j’ai eu l’occasion d’être associé à l'ensemble des étapes clés dans le développement de plusieurs projets scientifiques. Cette double compétence de biologie et de bioinformatique fait de moi un interlocuteur privilégié pour faire communiquer ces deux mondes. Cela me permet de contribuer significativement aux réflexions scientifiques, en particulier dans les domaines de l’infectiologie et la microbiologie. J'ai ainsi une vision plus moderne de l’étude des maladies infectieuses ce qui m’apporte une grande source d’innovation. C'est pourquoi je pense être à même de vous aider pour la mise en place de méthodes et l'exécution d’outils d'analyse de données de génomiques et transcriptomiques.

\end{cvletter}

%-------------------------------------------------------------------------------
% Preamble
% Main text
%\includegraphics[scale=0.08]{signature.jpg}

% Print the signature and enclosures with above letter informations
\makeletterclosing


\end{document}
